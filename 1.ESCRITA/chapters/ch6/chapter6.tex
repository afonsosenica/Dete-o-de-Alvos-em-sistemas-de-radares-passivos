% Chapter 6

\chapter{Conclusão} % Main chapter title
\label{chap:Chapter6} % For referencing the chapter elsewhere, use \ref{chap:Chapter6} 

%----------------------------------------------------------------------------------------
\section{Sumário}
O radar passivo é um sistema ótimo para contornar problemas como a deteção através de sistemas de contra-medidas e o elevado preço dos radares convencionais. Também é adequado para a deteção de alvos \textit{stealth} como já falado, devido à sua geometria bistática. No entanto, tudo tem um preço, e o radar passivo não é exceção. Devido ao número elevado de amostras e o processamento do sinal requerido para o tornar viável para a utilização no radar passivo, o sistema fica com um elevado custo computacional e consequentemente existe uma necessidade de processamento digital avançado. Posto isto, deve-se considerar o sistema de radar passivo como um complemento ao sistema de radar ativo, ao invés de uma substituição ao mesmo, visto que cada um colmata os pontos fracos do outro.\par 
Ao aproveitar sinais existentes no espetro eletromagnético, não existe mais poluição do mesmo, mas encontra-se um grande problema, que é o sinal não estar adaptado para a situação em particular tornando o processamento um processo muito mais complexo. Há poucos casos em que o iluminador pode estar adaptado, mas são em situações especificas como a utilização de satélites \gls{SAR} para a formação de imagem.\par 


\section{Discussão e Conclusões}
Seguindo o estudo efetuado durante o trabalho de investigação e do seu complemento com a atividade prática foram retiradas várias conclusões:


\begin{itemize}
\item A dessincronização dos dois canais em tempo é problemática no sentido em que diferença de fase entre o sinal direto e o sinal refletido provoca uma grande distorção em \textit{Doppler}, que é o que se pode verificar em todas as amostras retiradas. Este é um dos principais fatores para existir uma grande dispersão dos alvos em \textit{Doppler}, chegando a ir desde os $-1500 Hz$ aos $1500 Hz$, o que corresponde a uma velocidade de aproximadamente $2700 km/h$. De acordo com \cite{He2010} quanto maior for a velocidade, mais dispersão o sinal tem em \textit{Doppler};


\item A segunda conclusão, como já abordada no Capítulo \ref{chap:Chapter3}, é a pouca diretividade da antena utilizada. O seu diagrama de radiação (\ref{fig:yagi}) apresenta pouca diretividade no lóbulo principal e um lóbulos posterior com muita intensidade. Isto provoca a receção do sinal direto e do refletido na mesma antena o que degrada os resultados tornando-os menos fidedignos;


\item Como terceira conclusão, é importante abordar o tema da reconstrução e equalização do sinal direto, discutido no Capítulo \ref{chap:Chapter4}. A deteção utilizando um radar \gls{PCL} é baseada no cálculo de uma função de ambiguidade cruzada entre o sinal de direto e o refletido no objeto. Idealmente, o sinal recebido na antena de referência a apontar para o transmissor é uma cópia perfeita do sinal transmitido. No entanto, isto não acontece devido ao ruído introduzido e não temos acesso ao sinal direto a partir da localização do recetor, logo, este tem de ser obtido de outra forma. Uma abordagem a este constrangimento é recriar o sinal transmitido através da descodificação do sinal recebido e posterior codificação, obtendo assim uma cópia muito menos ruidosa do sinal transmitido. Ao não utilizar este método, é inserido muito efeito de \textit{multipath} o que prejudica a autenticidade do sinal.


\item Dos resultados do Capítulo \ref{chap:Chapter5} são feitas observações à relação entre o tempo de integração e a intensidade da correlação. De um modo geral, pode-se concluir que ao aumentar o tempo de integração e usando mais amostras, tem-se uma melhor relação sinal-ruído.


\item Ainda do Capítulo \ref{chap:Chapter5} são utilizadas várias amostras e zonas dessas amostras, ou seja, diferentes amostras de tempo do total da amostra. Ao utilizarmos amostras em diferentes marcas temporais somadas, obtemos melhores resultados que utilizando o mesmo número de amostras, mas apenas numa zona da matriz. Isto era de esperar visto que se cobre um tempo de integração maior com o mesmo número de custo computacional.


\item Um problema que afeta imenso a capacidade do radar é ter sido utilizada uma correlação simples. Como abordado no Capítulo \ref{chap:Chapter4}, o custo computacional de fazer uma correlação para milhões de amostras e sabendo que maior tempo de observação resulta em melhor resolução é muito elevado e não é suportável pelas máquinas a que temos acesso diariamente. Isto faz com que seja necessário a implementação de algoritmos que simplifiquem a correlação e contornem o problema do alto custo computacional. Ao não aplicar estes algoritmos os resultados ficam muito degradados, visto que se trabalha com muito menos amostras do que o ideal.


\end{itemize}


Em suma, o sistema de radar \gls{PCL} é muito vantajoso pelas demais razões já identificadas várias vezes e pode-se tornar um sistema muito mais económico, no entanto é necessário um processamento de sinal muito avançado e pesado. A não aplicação de certas técnicas compromete muito a operacionalidade e veracidade dos resultados obtidos pelo radar. Contudo, as conclusões retiradas deste trabalho de investigação permitem guiar futuros projetos e trabalhos de forma a escolher caminhos em que obtenham melhores resultados e permite saber o foco de trabalho para a resolução de determinados problemas.


\section{Cenários Possíveis - MARINHA}
“O projeto DESARMAR é uma iniciativa de investigação para o desenvolvimento de um sistema \gls{SAR} passivo baseado em \gls{SDR} a bordo de um sistema autónomo aéreo para monitorização e proteção do espaço litoral. Este projeto pretende investigar o potencial deste novo tipo de tecnologia e aplica-lo a vigilância e patrulhamento costeiro, fazendo uso de algoritmos de seguimento para identificar eventos de risco e ameaças de origem humana ou natural, permitindo assim a mitigação dos seus impactos (quer de um ponto de vista económico, quer social e ambiental)”. Este trabalho de investigação insere-se neste projeto de forma a contribuir para o estudo nas limitações da deteção utilizando um sistema de radar passivo. O projeto torna-se mais ambicioso com a utilização deste tipo de radares a bordo de um sistema autónomo aéreo e para a formação de imagem, contudo as lições aprendidas com esta dissertação podem direcionar o projeto num caminho mais eficaz.\par
A presente investigação é importante para a Marinha Portuguesa pelo conhecimento deste sistema, que pode ser uma nova realidade com inúmeras aplicações de interesse para a mesma. Um bom exemplo é o projeto \gls{SMARP}, que tem como objetivo projetar e construir uma demonstração de um radar passivo de matriz multi-banda baseado em \gls{SDR} com a finalidade de vigilância da costa.


\section{Propostas para Trabalhos Futuros}
Uma das principais conclusões retiradas desta investigação debruça-se sobre as poucas amostras utilizadas na correlação fazem com que o resultado seja muito degradado relativamente ao que podia ser. A primeira sugestão como complemento deste trabalho é a implementação de algoritmos discutidos ou não no Capítulo \ref{chap:Chapter4} e estudar a melhoria de resultados.\par 
Como segunda sugestão, é interessante realizar um estudo sobre a formação de imagem usando sistemas de radares passivos, o que vai de encontro às necessidades da Marinha e dos seus projetos como o projeto DESARMAR.