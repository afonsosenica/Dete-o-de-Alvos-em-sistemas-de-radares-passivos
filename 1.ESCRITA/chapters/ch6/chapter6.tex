% Chapter 6

\chapter{Conclusão} % Main chapter title
\label{chap:Chapter6} % For referencing the chapter elsewhere, use \ref{chap:Chapter6} 

%----------------------------------------------------------------------------------------
\section{Sumário}
O radar passivo é um sistema ótimo para contornar problemas como a deteção através de sistemas de contra-medidas e o elevado preço dos radares convencionais. Também é adequado para a deteção de alvos \textit{stealth} como já falado, devido à sua geometria bistática. No entanto, tudo tem um preço, e o radar passivo não é exceção. Devido ao número elevado de amostras e o processamento do sinal requerido para o tornar viável para a utilização no radar passivo, o sistema fica com um elevado custo computacional e consequentemente existe uma necessidade de processamento digital avançado. Posto isto, deve-se considerar o sistema de radar passivo como um complemento ao sistema de radar ativo, ao invés de uma substituição ao mesmo, visto que cada um colmata os pontos fracos do outro.\par 
Ao aproveitar sinais existentes no espetro eletromagnético, não existe mais poluição do mesmo, mas encontra-se um grande problema, que é o sinal não estar adaptado para a situação em particular tornando o processamento um processo muito mais complexo. Há poucos casos em que o iluminador pode estar adaptado, mas são em situações especificas como a utilização de satélites \gls{SAR} para a formação de imagem.


\section{Discussão e Conclusões}



\section{Cenários Possíveis - MARINHA}



\section{Propostas para Trabalhos Futuros}
Uma das principais conclusões retiradas desta investigação debruça-se sobre as poucas amostras utilizadas na correlação fazem com que o resultado seja muito degradado relativamente ao que podia ser. A primeira sugestão como complemento deste trabalho é a implementação de algoritmos discutidos ou não no Capítulo \ref{chap:Chapter4} e estudar a melhoria de resultados.\par 
Como segunda sugestão, é interessante realizar um estudo sobre a formação de imagem usando sistemas de radares passivos, o que vai de encontro às necessidades da Marinha e dos seus projetos como o projeto DESARMAR.