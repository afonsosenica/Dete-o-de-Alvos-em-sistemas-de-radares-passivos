% Capítulo 1
% 
\chapter{Introdução} % Título do capítulo
\label{chap:Chapter1} % Para fazer referência a esta secção ao longo da dissertação, use o comando Chapter~\ref{Chapter1}


%-------------------------------------------------------------------------------
%---------
%




\section{Sistemas Passivos para Deteção e Localização de Alvos}
A deteção e localização de alvos é feita de um modo convencional por um sistema de deteção ativo, ou seja, o radar convencional. De forma geral, existe um transmissor e um recetor, ambos controlados pelo operador que emite um sinal e é determinada uma distância através do tempo que este leva do recetor ao alvo e de volta ao transmissor.\par 
O radar passivo oferece a capacidade de detetar alvos usando iluminadores de oportunidade, conforme referido em \ref{IOS}. Isto permite detetar e localizar alvos como o radar ativo, com a vantagem operacional de não emitir nenhum sinal, o que se torna vantajoso não só pela razão mais óbvia em ambiente militar: a capacidade de não ser detetado; mas também outras, como a não poluição do espetro eletromagnético, utilizando assim sinais presentes neste.\par 
Por apresentar uma geometria bistática, como referido no Capítulo \ref{chap:Chapter2}, o radar \gls{PCL}, oferece a capacidade de detetar alvos \textit{stealth}, ou seja, alvos que apresentam um \textit{design} que tem como objetivo dispersar o sinal emitido do radar, por forma a não voltar ao mesmo local, o que é eficaz contra um radar convencional, mas no caso do radar bistático, apenas ajuda à sua deteção, aumentando a sua \gls{RCS}. Ainda é possível mencionar mais capacidades, como a melhor deteção de alvos a baixas altitudes e ainda a resistência a contra-medidas eletrónicas, ou seja, \textit{jamming}.\par
Dito isto, este tipo de sistemas providencia um leque de capacidades distinto do radar convencional e portanto não devem ser vistos como uma substituição deste, mas sim como um complemento do mesmo. 

\section{Sistemas de Radar Definidos por Software}
O sistema de radar passivo requer uma grande capacidade de computação e \textit{software} \parencite{Martorella}, o que implica a utilização de técnicas de processamento de modo a tornar o processo o mais eficiente possível. O Capítulo \ref{chap:Chapter4} refere-se ao processamento de sinal e deste é possível compreender porque o radar \gls{PCL} necessita de uma grande capacidade de computação, tanto como várias técnicas que são utilizadas.\par 
O conceito de sistema de radar definido por software vem do termo \gls{SDR}, que consiste num sistema de comunicações rádio onde as funções que seriam definidas por componentes físicos (\textit{hardware}), são agora definidas por \textit{software}. Usando um computador como exemplo, o \textit{hardware} consiste apenas na placa de som, o conversor analógico-digital e um circuito de receção de rádio. Isto permite uma grande flexibilidade e adaptabilidade da tecnologia.\par 
Quando um sistema de radar é desenhado, vários dos fatores mais relevantes são o desempenho e flexibilidade \parencite{Griffiths2017}. Com um radar definido por \textit{software} tem-se isto e muito mais. A sua aplicação de forma relativamente rápida o radar ao tipo de situação, seja para detetar veículos ou objetos de diferentes dimensões, estáticos ou com velocidades elevadas, ajustando frequências e técnicas no \textit{software}, são tudo processos que com o conhecimento de processamento de sinal necessário se tornam muito mais expeditos e baratos.\par 
Proporcionalmente ao avanço da tecnologia e ao rápido desenvolvimento da capacidade de processamento dos computadores, este tipo de radares vem a ser cada vez mais utilizado e com \textit{software} cada vez mais exigente e com mais capacidades.  

\section{Iluminadores de Oportunidade} \label{IOS}
O espetro eletromagnético encontra-se atualmente preenchido pelos mais diversos sinais, e cada vez mais com tendência para aumentar a ocupação deste. Portanto o sinal a utilizar, ou seja, o \gls{IO} pode ter as mais variadas caraterísticas. A sua escolha é crucial para o desempenho do sistema e de modo geral podem-se dividir em dois grandes grupos:
\begin{itemize}
\item Iluminadores de oportunidade terrestre;
\item Iluminadores de oportunidade espaciais.
\end{itemize}\par
No grupo dos \gls{IO}s terrestre encontram-se outros tipos de radares, como \gls{ATC}, sistemas de comunicações móveis como \gls{GSM} ou até \gls{WiFi}, e sistemas de \textit{broadcast}, como \gls{DAB}, transmissões \gls{FM} e \gls{DVB-T}.\par 
No grupo dos \gls{IO}s epaciais, encontram-se radares de monitorização da Terra, sistemas de televisão como \gls{DVB-S}, ou seja, televisão de satélite, sistemas de localização terrestre como \gls{GPS}, GLONASS e ainda sistemas de comunicações móveis por satélite como Iridium, Orbcomm e Globalstar.\par
Dentro dos parâmetros que mais influenciam a escolha do iluminador para o radar \gls{PCL} encontram-se a densidade de potência no alvo, a natureza da onda e a cobertura, por isso, devido à pouca cobertura, normalmente exclui-se logo outros radares como opção. No entanto, rádio \gls{FM} tem sido muito utilizado para este fim, especialmente nos primeiros anos deste século devido à sua elevada cobertura e densidade de potência, tanto como uma largura de banda aceitável e que depende do tipo de programa a ser transmitido que é um tópico abordado no Capítulo \ref{chap:Chapter5}. Este \gls{IO} tem vindo a ser substituído por outros serviços de \textit{broadcast} como \gls{DAB} e \gls{DVB-T} devido ao tópico falado anteriormente e consequentemente à necessidade de escolha de uma estação que transmita música mais adequada, o que também é uma variável que não se consegue controlar.


\begin{table}[h]
\centering
\begin{tabular}{@{}ccccc@{}}
\toprule
                    	 & FM                & DAB     & DVB-T     \\ \midrule
Banda de frequência      & 88 - 108$MHz$   & 174 - 240$MHz$  & 470 - 862$MHz$           \\
\textit{Network}    			 & MFN   & SFN   & SFN \\
Largura de banda de cada canal    & 150$kHz$ max.      & 1.536$MHz$  & 7.612$MHz$   \\ 
ERP típica($kW$)      & 2-250         & 0.5-10    & 1-100            \\ \bottomrule
\label{tab:cara}
\end{tabular}
\caption[Caraterísticas dos sinais FM, DAB e DVB-T]{Caraterísticas dos sinais FM, DAB e DVB-T \parencite{HeinerKuschel2019}}
\end{table} 
\par
A tabela 1.1, retirada do artigo \cite{HeinerKuschel2019}, apresenta as principais caraterísticas dos sinais mais utilizados como \gls{IO}s, com ERP:"\gls{ERP}", MFN:"\gls{MFN}" e SFN:"\gls{SFN}". Numa configuração \gls{SFN}, todos os transmissores da rede transmitem na mesma frequência, enquanto em \gls{MFN}, os transmissores transmitem em frequências diferentes. Na utilização de iluminadores em \gls{SFN}, o recetor do sinal direto recebe várias réplicas do sinal (\textit{multipath}) direto e o recetor do sinal refletido recebe várias réplicas do sinal refletido no alvo, o que provoca uma deteção com menor confiança.\par 
Comparando as outras caraterísticas, pode-se concluir que a \gls{DVB-T} tem maior largura de banda, o que se torna muito favorável para casos específicos como formação de imagem. É também importante referir que tanto a \gls{ERP} de \gls{FM} como \gls{DVB-T} permite ter uma boa densidade de potência no alvo a uma distância razoável quando comparado com o sinal \gls{DAB}.\par 
Para além da utilização destes iluminadores, também têm sido utilizados sinais \gls{GNSS} com sucesso no passado recente.

\section{Motivação e Objetivos}
Esta dissertação pretende abordar o estudo da deteção de alvos utilizando sistemas de radares Passivos. Dispositivos, no universo das tecnologias radar, com grande potencial de aplicação prática, não só no ambiente civil, como no militar.\par
Com a pesquisa, pretende-se efetuar o respetivo estado da arte, e aferir da sua pertinência de aplicação na Marinha, dotando-a de conhecimento, que nesta matéria lhe permita manter na vanguarda da evolução tecnológica. Tem também como objetivo, realizar o estudo sobre radares passivos de modo a, através de simulação de sinais \gls{DVB-T} e sinais \gls{FM}, simulação de antenas que garantam a adequada receção destes sinais, estudo das funções de ambiguidade de diversos sinais passíveis de serem utilizados como iluminadores de oportunidade e conseguir detetar alvos com este conceito. 

\section{Organização da Dissertação}
Esta dissertação divide-se em 6 capítulos, sendo o Capítulo \ref{chap:Chapter1} a introdução. No Capítulo \ref{chap:Chapter2} pretende-se introduzir o conceito de radar passivo, os conceitos a este adjacentes, a matemática básica que os suporta e uma breve introdução à formação de imagem. O Capítulo \ref{chap:Chapter3} é dedicado à teoria de antenas, passando pelos diversos tipos e os parâmetros fundamentais que as definem. No seguinte capítulo, pretende-se explicar a teoria do processamento de sinal num radar passivo de forma a haver deteção e pretende-se apresentar técnicas que o permitem fazer, o que leva ao Capítulo \ref{chap:Chapter5}, onde se pretende discutir como foi abordada a aplicação destes conceitos e as ferramentas que foram utilizadas. Por fim, o Capítulo \ref{chap:Chapter6} tem como objetivo a discussão de resultados obtidos e as conclusões a retirar da dissertação.\par 
É de referir que todos os documentos e programas relevantes feitos durante a dissertação, tanto como o documento da mesma estão disponíveis no \textit{GitHub}, na seguinte página de internet \url{https://github.com/afonsosenica/Tese}.\par 
Como suplemento do trabalho, existe um ficheiro na pasta \textit{GNU RADIO} que contem um programa que faz a receção de dados do LimeSDR ou de um \gls{SDR} com o objetivo de fazer a sua correlação. O \textit{GNU RADIO} foi o primeiro software que utilizado durante esta investigação, é bastante intuitivo e é uma boa ferramenta introdutória e permite retirar amostras que podem ser utilizadas em \textit{Octave} e \textit{Python} ou ainda analisar dados em tempo real dentro das limitações do recetor e do programa.