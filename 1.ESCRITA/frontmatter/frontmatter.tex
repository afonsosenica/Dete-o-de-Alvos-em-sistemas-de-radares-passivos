%O frontmmatter são as chamadas páginas iniciais, terá de atualizar as respetivas secções.

%-----------------------------------------%
%    CONFIGURAÇÃO INICIAL                 %
%-----------------------------------------%

%All acronyms must be written in this file.
\newacronym{RTS}{RTS}{Real-Time System}
\newacronym{UHF}{UHF}{Ultra High Frequency}

\frontmatter % Use roman page numbering style (i, ii, iii, iv...) for the pre-content pages

\pagestyle{plain} % Default to the plain heading style until the thesis style is called for the body content

%----------------------------------------------------------------------------------------
%	CAPA
%----------------------------------------------------------------------------------------
\maketitlepage


%----------------------------------------------------------------------------------------
%	CONTRA CAPA
%----------------------------------------------------------------------------------------
\makeconttitlepage


%-----------------------------------------%
%      EPÍGRAFE     (opcional)                     
%-----------------------------------------%
%\begin{epigraph}
%\null \vfill
%\begin{flushright}

%A epígrafe traduz-se pela inscrição de sentença conceituosa que, de algum modo inspirou o autor na elaboração do trabalho ou nas suas ações correntes, e que o mesmo considere importante revelar no trabalho. Tem uma natureza facultativa.

%\end{flushright}
%\vfill \null%
%\end{epigraph}%



%----------------------------------------------------------------------------------------
%	DEDICATÓRIA  (opcional)
%----------------------------------------------------------------------------------------
%\dedicatory{For/Dedicated to/To my\ldots}
\begin{dedicatory}
\null \vfill
\begin{flushright}

A dedicatória tem por finalidade prestar homenagem ou dedicar o trabalho a alguém próximo ou que tenha um especial significado para o autor do trabalho. 

É, também, um elemento facultativo na estrutura do trabalho, mas é usual que seja feita dedicando o trabalho aos pais, à família mais chegada ou a alguém com relevância especial na vida do autor. 

\end{flushright}
\vfill \null%
\end{dedicatory}


%----------------------------------------------------------------------------------------
%	AGRADECIMENTOS (opcional)
%----------------------------------------------------------------------------------------

\begin{acknowledgements}

Agradecimento é a expressão registada de uma gratidão às pessoas, entidades ou instituições que, de algum modo, contribuíram para a elaboração do trabalho. Sendo um elemento opcional, quando exista deve incluir-se na frente de folha a colocar logo após a folha de rosto ou das folhas da epígrafe e/ou da dedicatória, deixando o verso em branco.

\end{acknowledgements}




%----------------------------------------------------------------------------------------
%	RESUMO
%----------------------------------------------------------------------------------------

\begin{abstract}

%\noindent \textbf{\ Subtítulo caso queira!}

Desde o inicio da utilização de radares pelos militares que é conhecido o facto da vulnerabilidade da localização do transmissor quando se encontra a transmitir. Não só por este caso, mas também pela poluição do espetro eletromagnético ou pelo custo elevado de um transmissor, o radar passivo é uma solução ideal a todos estes problemas. No entanto, como tudo, tem as suas desvantagens, realçando não se controlar o sinal que é transmitido pelo iluminador de oportunidade e este não estar otimizado para sistemas de radar, o que no final, implica um processamento mais complexo.\par
Este conceito de radares passivos não é uma ideia recente. A primeira experiência realizada remonta ao ano de 1935, quando Robert Watson-Watt usou um iluminador de oportunidade de onda curta radiada do BBC Empire transmitter em Daventry para detetar um bombardeiro Heyford a uma distância de 8 km. No entanto, o primeiro radar passivo foi desenvolvido uns anos depois pelos alemães, denominado Klein Heidelberg.\par 
Esta dissertação tem como principal objetivo o desenvolvimento de um sistema de radar passivo, usando como iluminador de oportunidade, a televisão digital terrestre, \gls{DVB-T} e, simultaneamente, desenvolver um trabalho de pesquisa sobre radares passivos, processamento de sinal nos mesmos, teoria de antenas e formação de imagem utilizando radares passivos. Em jeito de conclusão e em função dos resultados obtidos pretende-se discutir possíveis cenários de implementação na Marinha Portuguesa.


% As palavras chave terão de ser definidas no ficheiro main.tex depois da linha de código keywords
\end{abstract}


%----------------------------------------------------------------------------------------
%	ABSTRACT 
%----------------------------------------------------------------------------------------
\begin{abstractotherlanguage}
% here you put the abstract in the "other language": English, if the work is written in Portuguese; Portuguese, if the work is written in English.


%\noindent \textbf{\ Subtitle if you want}
%\noindent 


The concept of passive radars is not a recent idea. In fact, the first experiment carried out dates back to the year of 1935 when Robert Watson-Watt used a BBC Empire transmitter shortwave illuminator of opportunity in Daventry to detect a Heyford bomber at a range of 8 km. However, the first passive radar was developed a few years later by the Germans, called Klein Heidelberg.\par 
This dissertation has as main objective the development of a passive radar system, using \gls{DVB-T} as an illuminator of opportunity and, simultaneously, to develop a research work on passive radars, its signal processing, basic theory of antennas and passive radars for image formation. As a conclusion and based on the results obtained, it is intended to discuss possible implementation scenarios in the Portuguese Navy.

% As palavras chave terão de ser definidas no ficheiro main.tex depois da linha de código conkeywords

\end{abstractotherlanguage}



%----------------------------------------------------------------------------------------
%	ÍNDICE DE CONTEÚDO / FIGURAS / TABELAS
%----------------------------------------------------------------------------------------

\tableofcontents % Imprime o índice principal
\pdfbookmark[0]{\contentsname}{toc}% Adiciona o índice aos bookmarks do pdf

\listoffigures % Imprime a lista de figuras

\listoftables % Imprime a lista de tabelas

\iflanguage{portuguese}{
\renewcommand{\listalgorithmname}{Lista de Algor\'itmos}
}
\listofalgorithms % Prints the list of algorithms
%\addchaptertocentry{\listalgorithmname} %Uncomment para mostrar no índice a lista de algoritmos


\renewcommand{\lstlistlistingname}{List of Source Code}
\iflanguage{portuguese}{
\renewcommand{\lstlistlistingname}{Lista de C\'odigo}
}
\lstlistoflistings % Imprime a lista de listagens (código-fonte da linguagem de programação)

%\addchaptertocentry{\lstlistlistingname} %Uncomment para mostrar a lista de de código no índice


%----------------------------------------------------------------------------------------
%	ABREVIATURAS
%----------------------------------------------------------------------------------------

\begin{abbreviations}{ll} % IncluI uma lista de abreviações (uma tabela de duas colunas)

%List of Abreviations
%
\textbf{UHF} & \textbf{U}ltra \textbf{H}igh \textbf{F}requency\\
\textbf{SINR} & \textbf{S}ignal to \textbf{I}nterference plus \textbf{N}oise \textbf{R}atio\\
\textbf{HPBW} & \textbf{H}alf \textbf{P}ower \textbf{B}eam\textbf{W}idth\\
\textbf{FNBW} & \textbf{F}irst \textbf{N}ull \textbf{B}eam\textbf{W}idth\\
\textbf{ROE} & \textbf{R}elação de \textbf{O}nda \textbf{E}stacionária\\
\textbf{PLF} & \textbf{P}olarization de \textbf{L}oss \textbf{F}actor\\

\end{abbreviations}

%----------------------------------------------------------------------------------------
%	SÍMBOLOS
%----------------------------------------------------------------------------------------

\begin{symbols}{lll} % Inclui uma lista de símbolos (uma tabela de três colunas)

%List of Symbols
%
$a$ & distance & \si{\meter} \\
$D$ & diretividade & \si{} \\
$D_{0}$ & diretividade máxima & \si{} \\
$D$ & Ganho & \si{} \\
$L$ & dimensão da antena & \si{\meter} \\
$P$ & potência & \si{\watt} \\
$r$ & raio & \si{\meter} \\
$U$ & intensidade de radiação & \si{\watt\per\steradian} \\
$U_{0}$ & intensidade de radiação isotrópica & \si{\watt\per\steradian} \\
$W$ & densidade de potência & \si{\watt\per\meter\squared} \\

%%Símbolo, nome e unidade

\addlinespace % espaçamento para separar os símbolos romanos dos grego

$\varphi$ & ângulo polar & \si{\radian} \\
$\sigma$ & radar cross section & \si{\meter\squared} \\
$\theta$ & azimute & \si{\radian} \\
$\lambda$ & comprimento de onda & \si{\meter} \\
$\omega$ & ângulo sólido & \si{\meter} \\



\end{symbols}



%----------------------------------------------------------------------------------------
%	ACRÓNIMOS
%----------------------------------------------------------------------------------------

%Use GLS
\glsresetall
\printglossary[title=\listacronymname,type=\acronymtype,style=long]


%----------------------------------------------------------------------------------------
%	ACABOU - BOM TRABALHO
%----------------------------------------------------------------------------------------

\mainmatter % Começar numeração da página com numéros árabes (1,2,3 ...)

