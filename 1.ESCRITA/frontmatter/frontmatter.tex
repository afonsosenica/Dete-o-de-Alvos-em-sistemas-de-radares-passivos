%O frontmmatter são as chamadas páginas iniciais, terá de atualizar as respetivas secções.

%-----------------------------------------%
%    CONFIGURAÇÃO INICIAL                 %
%-----------------------------------------%
%All acronyms must be written in this file.
\newacronym{DVB-T}{DVB-T}{Digital Video Broadcasting - Terrestrial}
\newacronym{FNBW}{FNBW}{First Null Beamwidth}
\newacronym{HPBW}{HPBW}{Half Power Beamwidth}
\newacronym{PLF}{PLF}{Polarization Loss Factor}
\newacronym{RCS}{RCS}{Radar Cross Section}
\newacronym{ROE}{ROE}{Relação de Onda Estacionária}
\newacronym{SINR}{SINR}{Signal to Interference Plus Noise Ratio}
\newacronym{UHF}{UHF}{Ultra High Frequency}

\frontmatter % Use roman page numbering style (i, ii, iii, iv...) for the pre-content pages

\pagestyle{plain} % Default to the plain heading style until the thesis style is called for the body content

%----------------------------------------------------------------------------------------
%	CAPA
%----------------------------------------------------------------------------------------
\maketitlepage


%----------------------------------------------------------------------------------------
%	CONTRA CAPA
%----------------------------------------------------------------------------------------
\makeconttitlepage


%-----------------------------------------%
%      EPÍGRAFE     (opcional)                     
%-----------------------------------------%
%\begin{epigraph}
%\null \vfill
%\begin{flushright}

%A epígrafe traduz-se pela inscrição de sentença conceituosa que, de algum modo inspirou o autor na elaboração do trabalho ou nas suas ações correntes, e que o mesmo considere importante revelar no trabalho. Tem uma natureza facultativa.

%\end{flushright}
%\vfill \null%
%\end{epigraph}%



%----------------------------------------------------------------------------------------
%	DEDICATÓRIA  (opcional)
%----------------------------------------------------------------------------------------
%\dedicatory{For/Dedicated to/To my\ldots}
\begin{dedicatory}
\null \vfill
\begin{flushright}

\textit{“All we have to decide is what to do with the time that is given us.”} — J.R.R. Tolkien

\end{flushright}
\vfill \null%
\end{dedicatory}


%----------------------------------------------------------------------------------------
%	AGRADECIMENTOS (opcional)
%----------------------------------------------------------------------------------------

\begin{acknowledgements}

Em primeiro lugar, um agradecimento muito especial para a minha família e aos meus amigos que me apoiaram incondicionalmente e acreditaram sempre em mim.\par
Ao meu orientador, Professor Doutor Paulo Marques por me ter aceitado neste desafio, pela sua paciência, disponibilidade fora de horas e motivação para que esta fase fosse concluída.\par
Ao meu co-orientador, CFR EN-AEL Fidalgo Neves pela disponibilidade em qualquer altura, pela preocupação, pela paciência na discussão de resultados e apoio. \par
À Sara, pela ajuda na atividade prática e pela paciência de ter que conviver com antenas durante uns bons meses dentro de casa.\par
Ao Araújo pela ajuda com os mais diversos problemas informáticos que me encontrei durante a dissertação.\par
À Marinha e à Escola Naval, pela disponibilidade e ajuda na aquisição de material.\par
A todas as pessoas que me ajudaram direta ou indiretamente na conclusão deste trabalho. A todos, o meu sincero obrigado pela ajuda e paciência infindável na conclusão desta etapa.

\end{acknowledgements}




%----------------------------------------------------------------------------------------
%	RESUMO
%----------------------------------------------------------------------------------------

\begin{abstract}

%\noindent \textbf{\ Subtítulo caso queira!}

Desde o inicio da utilização de radares pelos militares que é conhecido o facto da vulnerabilidade da localização do transmissor quando se encontra a transmitir. Não só por este caso, mas também pela poluição do espetro eletromagnético ou pelo custo elevado de um transmissor, o radar passivo é uma solução ideal a todos estes problemas. No entanto, como tudo, tem as suas desvantagens, realçando não se controlar o sinal que é transmitido pelo iluminador de oportunidade e este não estar otimizado para sistemas de radar, o que no final, implica um processamento mais complexo.\par
Este conceito de radares passivos não é uma ideia recente. A primeira experiência realizada remonta ao ano de 1935, quando Robert Watson-Watt usou um iluminador de oportunidade de onda curta radiada do BBC Empire transmitter em Daventry para detetar um bombardeiro Heyford a uma distância de 8 km. No entanto, o primeiro radar passivo foi desenvolvido uns anos depois pelos alemães, denominado Klein Heidelberg.\par 
Esta dissertação tem como principal objetivo o desenvolvimento e estudo de um sistema de radar passivo para a deteção de alvos, usando como iluminador de oportunidade, a televisão digital terrestre, \gls{DVB-T} e, simultaneamente, desenvolver um trabalho de pesquisa sobre radares passivos, processamento de sinal nos mesmos, teoria de antenas e formação de imagem utilizando radares passivos. Em jeito de conclusão e em função dos resultados obtidos pretende-se discutir possíveis cenários de implementação na Marinha Portuguesa.


% As palavras chave terão de ser definidas no ficheiro main.tex depois da linha de código keywords
\end{abstract}


%----------------------------------------------------------------------------------------
%	ABSTRACT 
%----------------------------------------------------------------------------------------
\begin{abstractotherlanguage}
% here you put the abstract in the "other language": English, if the work is written in Portuguese; Portuguese, if the work is written in English.


%\noindent \textbf{\ Subtitle if you want}
%\noindent 

Since the beginning of the use of radars by the military it is known the fact of vulnerability in the location of the transmitter when it is operating. It is not only by this specific reason, but also because of the pollution of the electromagnetic spectrum or the high cost of a transmitter, the passive radar is an ideal solution to all these problems. However, like everything, it has its disadvantages, such as the signal that is transmitted by the illuminator of opportunity is not controlled and it is not optimized for radar systems, which in the end, implies a more complex processing.
The concept of passive radars is not a recent idea. In fact, the first experiment carried out dates back to the year of 1935 when Robert Watson-Watt used a BBC Empire transmitter shortwave illuminator of opportunity in Daventry to detect a Heyford bomber at a range of 8 km. However, the first passive radar was developed a few years later by the Germans, called Klein Heidelberg.\par 
This dissertation has as main objective the development of a passive radar system, using \gls{DVB-T} as an illuminator of opportunity and, simultaneously, to develop a research work on passive radars, its signal processing, basic theory of antennas and passive radars for image formation. As a conclusion and based on the results obtained, it is intended to discuss possible implementation scenarios in the Portuguese Navy.

% As palavras chave terão de ser definidas no ficheiro main.tex depois da linha de código conkeywords

\end{abstractotherlanguage}



%----------------------------------------------------------------------------------------
%	ÍNDICE DE CONTEÚDO / FIGURAS / TABELAS
%----------------------------------------------------------------------------------------

\tableofcontents % Imprime o índice principal
\pdfbookmark[0]{\contentsname}{toc}% Adiciona o índice aos bookmarks do pdf

\listoffigures % Imprime a lista de figuras

\listoftables % Imprime a lista de tabelas

\iflanguage{portuguese}{
\renewcommand{\listalgorithmname}{Lista de Algor\'itmos}
}
\listofalgorithms % Prints the list of algorithms
%\addchaptertocentry{\listalgorithmname} %Uncomment para mostrar no índice a lista de algoritmos


\renewcommand{\lstlistlistingname}{List of Source Code}
\iflanguage{portuguese}{
\renewcommand{\lstlistlistingname}{Lista de C\'odigo}
}
\lstlistoflistings % Imprime a lista de listagens (código-fonte da linguagem de programação)

%\addchaptertocentry{\lstlistlistingname} %Uncomment para mostrar a lista de de código no índice


%----------------------------------------------------------------------------------------
%	ABREVIATURAS
%----------------------------------------------------------------------------------------

%\begin{abbreviations}{ll} % IncluI uma lista de abreviações (uma tabela de duas colunas)
%
%%List of Abreviations
%
\textbf{2D-CCF} & 2-\textbf{D}imensional \textbf{C}ross-\textbf{C}orrelation \textbf{F}unction\\
\textbf{CCF} & \textbf{C}ross-\textbf{C}orrelation \textbf{F}unction\\
\textbf{CNIT} & \textbf{I}talian \textbf{N}ational \textbf{C}onsortium for \textbf{T}elecommunications \\
\textbf{CPI} & \textbf{C}oherent \textbf{P}rocessing \textbf{I}nterval\\
\textbf{DFT} & \textbf{D}irect \textbf{F}ourier \textbf{T}ransform\\
\textbf{DVB} & \textbf{D}igital \textbf{V}idep \textbf{B}roadcasting\\
\textbf{DVB-T} & \textbf{D}igital \textbf{V}idep \textbf{B}roadcasting - \textbf{T}errestrial\\
\textbf{FNBW} & \textbf{F}irst \textbf{N}ull \textbf{B}eam\textbf{W}idth\\
\textbf{GNSS} & \textbf{G}lobal \textbf{N}avigation \textbf{S}atellite \textbf{S}ystem\\
\textbf{GPS} & \textbf{G}lobal \textbf{P}ositioning \textbf{S}ystem\\
\textbf{HPBW} & \textbf{H}alf \textbf{P}ower \textbf{B}eam\textbf{W}idth\\
\textbf{IDFT} & \textbf{I}nverse \textbf{D}irect \textbf{F}ourier \textbf{T}ransform\\
\textbf{PLF} & \textbf{P}olarization de \textbf{L}oss \textbf{F}actor\\
\textbf{PCL} & \textbf{P}assive  \textbf{C}oherent \textbf{L}ocation\\
\textbf{RCS} & \textbf{R}adar \textbf{C}ross \textbf{S}ection\\
\textbf{ROE} & \textbf{R}elação de \textbf{O}nda \textbf{E}stacionária\\
\textbf{SINR} & \textbf{S}ignal to \textbf{I}nterference plus \textbf{N}oise \textbf{R}atio\\
\textbf{SNR} & \textbf{S}ignal to \textbf{N}oise \textbf{R}atio\\
\textbf{UHF} & \textbf{U}ltra \textbf{H}igh \textbf{F}requency\\
\textbf{VHF} & \textbf{V}ery \textbf{H}igh \textbf{F}requency\\

%
%\end{abbreviations}

%----------------------------------------------------------------------------------------
%	SÍMBOLOS
%----------------------------------------------------------------------------------------

\begin{symbols}{lll} % Inclui uma lista de símbolos (uma tabela de três colunas)

%List of Symbols
%
$a$ & distance & \si{\meter} \\
$D$ & dimensão da antena & \si{\meter} \\
$P$ & power & \si{\watt} (\si{\joule\per\second}) \\
$r$ & raio & \si{\meter} \\

%%Símbolo, nome e unidade

\addlinespace % espaçamento para separar os símbolos romanos dos grego

$\varphi$ & ângulo polar & \si{\radian} \\
$\theta$ & azimute & \si{\radian} \\
$\lambda$ & comprimento de onda & \si{\meter} \\


\end{symbols}



%----------------------------------------------------------------------------------------
%	ACRÓNIMOS
%----------------------------------------------------------------------------------------


%Use GLS
\glsresetall
\printglossary[title=\listacronymname,type=\acronymtype,style=long]


%----------------------------------------------------------------------------------------
%	ACABOU - BOM TRABALHO
%----------------------------------------------------------------------------------------

\mainmatter % Começar numeração da página com numéros árabes (1,2,3 ...)

