% Capítulo 1
% 
\chapter{Introdução} % Título do capítulo
\label{chap:Chapter1} % Para fazer referência a esta secção ao longo da dissertação, use o comando Chapter~\ref{Chapter1}


%-------------------------------------------------------------------------------
%---------
%




\section{Sistemas Passivos para Deteção e Localização de Alvos}
Os radares convencionais apresentam uma configuração onde são constituídos por um transmissor e um recetor. Estes podem estar no mesmo local, apresentando uma geometria mono-estática, ou em locais diferentes, apresentando uma geometria bi-estática. Neste tipo de radares, um pulso é transmitido em forma de energia eletromagnética, e através do conhecimento do tempo levado pelo pulso a ser transmitido e recebido depois de refletido no alvo e da velocidade de propagação da luz, consegue-se determinar um valor de distância.\par 
Num radar passivo, não existe transmissão de energia eletromagnética durante o seu funcionamento. Ao invés, utiliza iluminadores de oportunidade e compara o seu sinal direto com pequenas alterações que ocorrem no campo eletromagnético por alvos em movimento de forma a detetar um alvo \parencite{Griffiths2017}.\par 
Este sistema radar pode utilizar uma grande variedade de iluminadores, desde sistemas de navegação por satélite (\textit{\gls{GNSS}}) como o \gls{GPS} ou o GLONASS, \textit{routers} de WiFi ou qualquer sistema de transmissão de frequências rádio como \textit{\gls{DVB}} ou estações de rádio. Dito isto, por forma a dimensionar o sistema para o efeito desejado, torna-se necessário uma boa compreensão das mais diversas caraterísticas dos iluminadores, especialmente a densidade de potência no alvo, a cobertura e a natureza da onda.\par 
Para a finalidade de deteção de alvos a grandes distâncias, os sinais mais eficazes e consequentemente mais utilizados são os que apresentam elevada potência, como transmissores de \gls{VHF} e de televisão digital em \gls{UHF}, não obstante poder-te também utilizar em certos casos outros iluminadores.

\section{Sistemas de Radar Definidos por Software}


\section{Motivação e Objetivos}


\section{Organização da Dissertação}