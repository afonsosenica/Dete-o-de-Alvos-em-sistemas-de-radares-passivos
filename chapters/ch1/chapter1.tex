% Capítulo 1
% 
\chapter{Introdução} % Título do capítulo
\label{chap:Chapter1} % Para fazer referência a esta secção ao longo da dissertação, use o comando Chapter~\ref{Chapter1}


%-------------------------------------------------------------------------------
%---------
%




\section{Sistemas Passivos para Deteção e Localização de Alvos}


1. Introdução 

1.1 Sistemas passivos para deteção e localização de alvos 

1.1.1 Sub-subcapítulo 

etc...

Aos capítulos e subcapítulos devem ser dados títulos, em letra destacada em negrito, de corpo sucessivamente 14, 13 e 12, sempre encostados à margem esquerda da página sem qualquer avanço.

Não é possível apresentar um critério único para o ordenamento de capítulos e subcapítulos, decorrendo esta estrutura da natureza do próprio trabalho, variando consoante a área disciplinar ou científica do mesmo e das suas características próprias.\\
Nalguns casos terá uma natureza explicativa, noutros passará pela exposição de resultados e sua interpretação, envolvendo a apresentação de critérios, tabelas de resultados, memória descritiva, etc.

Cada um dos capítulos deve começar ao cimo de uma página ímpar (à direita).

\section{Sistemas de Radar Definidos por Software}


\section{Motivação e Objetivos}


\section{Organização da Dissertação}