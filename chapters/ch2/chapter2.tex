% Chapter 2

\chapter{Radares Passivos} % Main chapter title

\label{chap:Chapter2} % For referencing the chapter elsewhere, use \ref{chap:Chapter2} 

%----------------------------------------------------------------------------------------

\section{Contextualização}
Os radares convencionais apresentam uma configuração onde são constituídos por um transmissor e um recetor, normalmente no mesmo local. Neste tipo de radares, um pulso é transmitido em forma de energia eletromagnética, e através do conhecimento do tempo levado pelo pulso a ser transmitido e recebido depois de refletido no alvo e da velocidade de propagação da luz, consegue-se determinar um valor de distância.\par 
Num radar passivo, não existe transmissão de energia eletromagnética durante o seu funcionamento. Ao invés, utiliza iluminadores de oportunidade e compara o seu sinal direto com pequenas alterações que ocorrem no campo eletromagnético por alvos em movimento de forma a detetar um alvo \parencite{Griffiths2017}.\par 
Este sistema radar pode utilizar uma grande variedade de iluminadores, desde sistemas de navegação por satélite (\textit{\gls{GNSS}}) como o \gls{GPS} ou o GLONASS, \textit{routers} de WiFi ou qualquer sistema de transmissão de frequências rádio como \textit{\gls{DVB}} ou estações de rádio. Dito isto, por forma a dimensionar o sistema para o efeito desejado, torna-se necessário uma boa compreensão das mais diversas caraterísticas dos iluminadores, como é falado mais à frente neste capítulo.\par 
Para a finalidade de deteção de alvos a grandes distâncias, os sinais mais eficazes e consequentemente mais utilizados são os que apresentam elevada potência, como transmissores de \gls{VHF} e de televisão digital em \gls{UHF}, não obstante poder-se também utilizar em certos casos outros iluminadores.\par
O cenário típico de um esquema de deteção usando um radar passivo é, como mostrado na Figura \ref{fig:esquema_pcl}, constituído por duas antenas recetoras, uma antena que recebe o sinal direto do iluminador ($S_{ref}$) e outra antena que recebe o sinal que é refletido no alvo ($S_{r}$). O sinal refletido no alvo fornece duas informações importantes para a sua deteção: o \textit{bistatic range}, ou seja, a distância ao alvo, conseguida através da diferença de tempo entre o sinal direto e o sinal refletido; e o \textit{Doppler}, que é o desvio de frequência que um alvo em movimento cria no sinal que é refletido devido à sua velocidade. Estes conceitos são discutidos mais à frente neste capitulo. \par

\begin{figure}[h]
\centering
\includegraphics[scale=0.7]{chapters/ch2/assets/esquema_pcl}
\caption[Esquema Geometria Radar Passivo]{Esquema da geometria de um radar passivo}
\label{fig:esquema_pcl}
\end{figure}

O conceito do radar passivo é fazer uma relação cruzada, ou, como mais conhecido o termo, \textit{cross-correlation} entre o sinal direto e o sinal refletido em função das variáveis \textit{delay-time} que pode ser transformado em \textit{bistatic range} e o desvio de \textit{Doppler}. A \textit{cross-correlation}, de forma simples, é uma medida de similaridade entre dois sinais aplicando um atraso num deles, que neste caso, para além do atraso (\textit{delay-time}), também é feita para os diferentes \textit{Doppler}, ou seja, em duas dimensões. No entanto, na prática existem processos analíticos mais eficientes, visto que fazer a \textit{cross-correlation} a duas dimensões em tempo real torna o processo muito pesado computacionalmente.



\subsection{Geometrias Radar}
Podemos classificar os radares quanto à localização dos transmissores e recetores. O ângulo $\beta$ que estes formam, sendo o seu centro o alvo, determina o tipo de geometria \parencite{Baker2019}. Se $\beta <20^{\circ}$, o transmissor e o recetor encontram-se perto ou no mesmo sítio, então estamos perante uma geometria monostática (Figura \ref{fig:monostatic}). Quando o transmissor e recetor estão mais afastados e formam um ângulo com centro no recetor dentro dos seguintes limites, $20^{\circ}<\beta <145^{\circ}$, a geometria é bistática (Figura \ref{fig:bistatic}). Para situações particulares, em que o alvo se encontra a uma cota baixa em relação à linha imaginária que une o transmissor e o recetor ($145^{\circ}<\beta <180^{\circ}$), estamos perante uma geometria \textit{Forward Scatter} (Figura \ref{fig:fsc}).\par   

\begin{figure}[h]
\centering
\includegraphics[scale=0.8]{chapters/ch2/assets/monostatic}
\caption[Geometria Monostática]{Geometria Monostática}
\label{fig:monostatic}
\end{figure}

\begin{figure}[h]
\centering
\includegraphics[scale=0.8]{chapters/ch2/assets/bistatic}
\caption[Geometria Bistática]{Geometria Bistática}
\label{fig:bistatic}
\end{figure}

\begin{figure}[h]
\centering
\includegraphics[scale=0.7]{chapters/ch2/assets/fsc}
\caption[Geometria \textit{Forward Scatter}]{Geometria \textit{Forward Scatter}}
\label{fig:fsc}
\end{figure}

Os radares passivos, como já discutido, têm a vantagem de não transmitirem um sinal, e ao invés usar um sinal a ser transmitido por outra fonte. Isto implica que o transmissor e o recetor não estejam no mesmo sítio nem perto, logo, quando se fala em radares passivos, assume-se uma geometria bistática.



\subsection{Alcance Bistático e \textit{Doppler}}
Como falado no ínicio deste capítulo, o alcance bistático, ou \textit{bistatic range} e o desvio de \textit{Doppler} são varáveis fundamentais para qualquer sistema radar e isso não exclui o radar passivo.\par 
O recetor bistático pode medir 3 parâmetros diferentes:
\begin{itemize}
\item A diferença em alcance entre o sinal direto e o sinal que é refletido, ou seja, o \textit{bistatic range};
\item O desvio de \textit{Doppler} do sinal recebido;
\item O ângulo $\theta_{R}$ do sinal recebido, se for usada uma antena de \textit{surveillance }direcional.
\end{itemize}

\subsubsection*{Alcance Bistático} 
Tal como representado na Figura \ref{fig:geom}, tomamos os valores $R_{T}$ como a distância do transmissor ao alvo,  $R_{R}$ como a distância do recetor ao alvo,  $\beta$ como o ângulo entre estes e com centro no alvo, e  $C$ como a distância do transmissor ao recetor, ou, \textit{Baseline}.\par


\begin{figure}[h]
\centering
\includegraphics[scale=0.9]{chapters/ch2/assets/geom}
\caption[Parâmetros na geometria bistática]{Parâmetros na geometria bistática}
\label{fig:geom}
\end{figure}

O termo alcance bistático, ou \textit{bistatic range}, é definido em \ref{2.1}. Com este valor é possível criar elipses bistáticas (para duas dimensões) ou elipsoides bistáticos (para três dimensões) com o transmissor e o recetor como dois focos das mesmas. \par

\begin{equation} \label{2.1}
R_{T}+R_{R}-C
\end{equation}

Contudo, se a \textit{baseline} $C$ for um valor conhecido, pode-se extrair o termo \textit{range sum} $R_{T}+R_{R}$. \par
Através do conhecimento do valor de $\theta_{R}$, que é mensurável se a antena de \textit{surveillance} for direcional, a distância do alvo ao recetor é dada pela expressão \ref{2.2}.

\begin{equation} \label{2.2}
R_{R}=\dfrac{\left(  R_{T}+R_{R}\right)^{2}-C^{2}}{2\left(  R_{T}+R_{R}+C sin\theta_{R}\right)}
\end{equation}


Um dos parâmetros importantes quando se fala em alcance bistático é a \textit{range resolution}, ou seja, a resolução em alcance. Este parâmetro é definido pela capacidade de distinguir os alvos que estão muito próximos. Um bom exemplo de um sistema radar que necessite de grande \textit{range resolution} é um sistema de direção de tiro. \par 

Num radar convencional monostático, a resolução em alcance é dada por $\Delta R=c/2B$, onde c é a velocidade de propagação e B a largura de banda do sinal transmitido. No entanto, num radar passivo, a geometria é bistática, o que leva a existirem diferentes elipses bistáticas concêntricas, isto é, com centro no mesmo ponto, o que tem de ser tomado em conta na expressão que representa a \textit{range resolution}:

\begin{equation} \label{2.3}
\Delta r=\dfrac{c}{2B\left( \dfrac{cos\beta}{2}\right)}
\end{equation}

No entanto, este caso é específico para quando os dois alvos estão alinhados relativamente à bissetriz do ângulo $\beta$, como é possivel observar na figura \ref{fig:geom_varios_alvos} o exemplo dos alvos 1 e 2. Para um caso generalizado, como por exemplo o alvo 1 e o alvo 3, a expressão da \textit{bistatic range resolution} (Expressão \ref{2.4}) depende de mais um valor $\varphi$ representado na figura \ref{fig:geom_varios_alvos} como o ângulo entre o seguimento da bissetriz do ângulo $\beta$ e o segmento de reta que une o alvo 1 e o alvo 3 com centro no alvo 1. 

\begin{figure}[h]
\centering
\includegraphics[scale=0.5]{chapters/ch2/assets/geom_varios_alvos}
\caption[Geometria bistática para vários alvos]{Geometria bistática para vários alvos (Adaptada da figura 2.4 \cite{Griffiths2017})}
\label{fig:geom_varios_alvos}
\end{figure}


\begin{equation} \label{2.4}
\Delta r=\dfrac{c}{\left[ 2B\left( \dfrac{cos\beta}{2}\right)\right] cos\varphi}
\end{equation}


A expressão do \textit{bistatic range resolution} permite interpretar a geometria bistática quanto à distância entre o transmissor e recetor. Da expressão \ref{2.4} conclui-se que quanto mais o ângulo $\beta$ se aproxima de um ângulo reto, o denominador tende para um valor próximo de 0, ou seja, a resolução em alcance torna-se fraca. Contudo, nesta situação estamos perante uma geometria \textit{forward scatter}, discutido no inicio deste capítulo, o que pode ser contornando usando vários recetores em locais diferentes.\par 
Para radares passivos, continuando a interpretação da expressão \ref{2.4}, os iluminadores de oportunidade mais utilizados têm pouca largura de banda $B$, o que se reflete numa resolução em alcance mais reduzida. No entanto, os sinais de DVB-T, discutidos no Capítulo \ref{chap:Chapter1}, têm uma largura de banda na ordem dos 8 MHz, o que já permite uma resolução em alcance na ordem dos 40m.

\subsubsection*{\textit{Doppler}} 
O desvio de Doppler ocorre quando pelo menos um dos elementos transmissor, alvo, recetor se encontra em movimento. No caso mais comum, em que apenas o alvo se encontra em movimento, o desvio de \textit{Doppler} é dado por \parencite{Griffiths2017},

\begin{equation} \label{2.5}
f_{D}=\dfrac{2v}{\lambda}cos\delta cos\left( \dfrac{\beta}{2}\right) 
\end{equation}

onde $\delta$ é o ângulo formado pelo sentido do vetor velocidade $v$ e a bissetriz do ângulo $\beta$ com centro no alvo.




\subsection{Formação de Imagem}

