%O frontmmatter são as chamadas páginas iniciais, terá de atualizar as respetivas secções.

%-----------------------------------------%
%    CONFIGURAÇÃO INICIAL                 %
%-----------------------------------------%

%All acronyms must be written in this file.
\newacronym{DVB-T}{DVB-T}{Digital Video Broadcasting - Terrestrial}
\newacronym{FNBW}{FNBW}{First Null Beamwidth}
\newacronym{HPBW}{HPBW}{Half Power Beamwidth}
\newacronym{PLF}{PLF}{Polarization Loss Factor}
\newacronym{RCS}{RCS}{Radar Cross Section}
\newacronym{ROE}{ROE}{Relação de Onda Estacionária}
\newacronym{SINR}{SINR}{Signal to Interference Plus Noise Ratio}
\newacronym{UHF}{UHF}{Ultra High Frequency}

\frontmatter % Use roman page numbering style (i, ii, iii, iv...) for the pre-content pages

\pagestyle{plain} % Default to the plain heading style until the thesis style is called for the body content

%----------------------------------------------------------------------------------------
%	CAPA
%----------------------------------------------------------------------------------------
\maketitlepage


%----------------------------------------------------------------------------------------
%	CONTRA CAPA
%----------------------------------------------------------------------------------------
\makeconttitlepage


%-----------------------------------------%
%      EPÍGRAFE     (opcional)                     
%-----------------------------------------%
%\begin{epigraph}
%\null \vfill
%\begin{flushright}

%A epígrafe traduz-se pela inscrição de sentença conceituosa que, de algum modo inspirou o autor na elaboração do trabalho ou nas suas ações correntes, e que o mesmo considere importante revelar no trabalho. Tem uma natureza facultativa.

%\end{flushright}
%\vfill \null%
%\end{epigraph}%



%----------------------------------------------------------------------------------------
%	DEDICATÓRIA  (opcional)
%----------------------------------------------------------------------------------------
%\dedicatory{For/Dedicated to/To my\ldots}
\begin{dedicatory}
\null \vfill
\begin{flushright}

A dedicatória tem por finalidade prestar homenagem ou dedicar o trabalho a alguém próximo ou que tenha um especial significado para o autor do trabalho. 

É, também, um elemento facultativo na estrutura do trabalho, mas é usual que seja feita dedicando o trabalho aos pais, à família mais chegada ou a alguém com relevância especial na vida do autor. 

\end{flushright}
\vfill \null%
\end{dedicatory}


%----------------------------------------------------------------------------------------
%	AGRADECIMENTOS (opcional)
%----------------------------------------------------------------------------------------

\begin{acknowledgements}

Agradecimento é a expressão registada de uma gratidão às pessoas, entidades ou instituições que, de algum modo, contribuíram para a elaboração do trabalho. Sendo um elemento opcional, quando exista deve incluir-se na frente de folha a colocar logo após a folha de rosto ou das folhas da epígrafe e/ou da dedicatória, deixando o verso em branco.

\end{acknowledgements}




%----------------------------------------------------------------------------------------
%	RESUMO
%----------------------------------------------------------------------------------------

\begin{abstract}

%\noindent \textbf{\ Subtítulo caso queira!}

[Segue-se, com caráter obrigatório, um resumo em língua portuguesa e em língua inglesa (abstract), cada um deles com um máximo de 300 palavras.]

Os radares convencionais apresentam uma configuração onde são constituídos por um transmissor e um recetor. Estes podem estar no mesmo local, apresentando uma geometria mono-estática, ou em locais diferentes, apresentando uma geometria bi-estática. Neste tipo de radares, um pulso é transmitido em forma de energia eletromagnética, e através do conhecimento do tempo levado pelo pulso a ser transmitido e recebido depois de refletido no alvo e da velocidade de propagação da luz, consegue-se determinar um valor de distância.\par 
Num radar passivo, não existe transmissão de energia eletromagnética durante o seu funcionamento. Ao invés, utiliza iluminadores de oportunidade e compara o seu sinal direto com pequenas alterações que ocorrem no campo eletromagnético por alvos em movimento de forma a detetar um alvo \parencite{Griffiths2017}.\par 
O conceito de radares passivos não é uma ideia recente. A primeira experiência realizada remonta ao ano de 1935, quando \textit{Robert Watson-Watt} usou um iluminador de oportunidade de onda curta radiada do \textit{BBC Empire transmitter} em Daventry para detetar um bombardeiro \textit{Heyford} a uma distância de 8 km. No entanto, o primeiro radar passivo foi desenvolvido uns anos depois pelos alemães, denominado \textit{Klein Heidelberg}.\par 
Esta dissertação tem como principal objetivo o desenvolvimento de um sistema de radar passivo, usando como iluminador de oportunidade, a televisão digital terrestre, \gls{DVB-T}


% As palavras chave terão de ser definidas no ficheiro main.tex depois da linha de código keywords
\end{abstract}


%----------------------------------------------------------------------------------------
%	ABSTRACT 
%----------------------------------------------------------------------------------------
\begin{abstractotherlanguage}
% here you put the abstract in the "other language": English, if the work is written in Portuguese; Portuguese, if the work is written in English.


%\noindent \textbf{\ Subtitle if you want}
%\noindent 

Trabalhos escritos em língua Inglesa devem incluir um resumo alargado com cerca de 1000 palavras, ou duas páginas.

Se o trabalho estivesse escrito em Português, este resumo seria em língua Inglesa, com cerca de 200 palavras, ou uma página.

Para alterar a língua basta ir às configurações do documento no ficheiro \file{main.tex} e alterar para a língua desejada ('english' ou 'portuguese')\footnote{Alterar a língua requer apagar alguns ficheiros temporários; O target \keyword{clean} do \keyword{Makefile} incluído pode ser utilizado para este propósito.}. Isto fará com que os cabeçalhos incluídos no template sejam traduzidos para a respetiva língua.

% As palavras chave terão de ser definidas no ficheiro main.tex depois da linha de código conkeywords

\end{abstractotherlanguage}



%----------------------------------------------------------------------------------------
%	ÍNDICE DE CONTEÚDO / FIGURAS / TABELAS
%----------------------------------------------------------------------------------------

\tableofcontents % Imprime o índice principal
\pdfbookmark[0]{\contentsname}{toc}% Adiciona o índice aos bookmarks do pdf

\listoffigures % Imprime a lista de figuras

\listoftables % Imprime a lista de tabelas

\iflanguage{portuguese}{
\renewcommand{\listalgorithmname}{Lista de Algor\'itmos}
}
\listofalgorithms % Prints the list of algorithms
%\addchaptertocentry{\listalgorithmname} %Uncomment para mostrar no índice a lista de algoritmos


\renewcommand{\lstlistlistingname}{List of Source Code}
\iflanguage{portuguese}{
\renewcommand{\lstlistlistingname}{Lista de C\'odigo}
}
\lstlistoflistings % Imprime a lista de listagens (código-fonte da linguagem de programação)

%\addchaptertocentry{\lstlistlistingname} %Uncomment para mostrar a lista de de código no índice


%----------------------------------------------------------------------------------------
%	ABREVIATURAS
%----------------------------------------------------------------------------------------

\begin{abbreviations}{ll} % IncluI uma lista de abreviações (uma tabela de duas colunas)

%List of Abreviations
%
\textbf{2D-CCF} & 2-\textbf{D}imensional \textbf{C}ross-\textbf{C}orrelation \textbf{F}unction\\
\textbf{CCF} & \textbf{C}ross-\textbf{C}orrelation \textbf{F}unction\\
\textbf{CNIT} & \textbf{I}talian \textbf{N}ational \textbf{C}onsortium for \textbf{T}elecommunications \\
\textbf{CPI} & \textbf{C}oherent \textbf{P}rocessing \textbf{I}nterval\\
\textbf{DFT} & \textbf{D}irect \textbf{F}ourier \textbf{T}ransform\\
\textbf{DVB} & \textbf{D}igital \textbf{V}idep \textbf{B}roadcasting\\
\textbf{DVB-T} & \textbf{D}igital \textbf{V}idep \textbf{B}roadcasting - \textbf{T}errestrial\\
\textbf{FNBW} & \textbf{F}irst \textbf{N}ull \textbf{B}eam\textbf{W}idth\\
\textbf{GNSS} & \textbf{G}lobal \textbf{N}avigation \textbf{S}atellite \textbf{S}ystem\\
\textbf{GPS} & \textbf{G}lobal \textbf{P}ositioning \textbf{S}ystem\\
\textbf{HPBW} & \textbf{H}alf \textbf{P}ower \textbf{B}eam\textbf{W}idth\\
\textbf{IDFT} & \textbf{I}nverse \textbf{D}irect \textbf{F}ourier \textbf{T}ransform\\
\textbf{PLF} & \textbf{P}olarization de \textbf{L}oss \textbf{F}actor\\
\textbf{PCL} & \textbf{P}assive  \textbf{C}oherent \textbf{L}ocation\\
\textbf{RCS} & \textbf{R}adar \textbf{C}ross \textbf{S}ection\\
\textbf{ROE} & \textbf{R}elação de \textbf{O}nda \textbf{E}stacionária\\
\textbf{SINR} & \textbf{S}ignal to \textbf{I}nterference plus \textbf{N}oise \textbf{R}atio\\
\textbf{SNR} & \textbf{S}ignal to \textbf{N}oise \textbf{R}atio\\
\textbf{UHF} & \textbf{U}ltra \textbf{H}igh \textbf{F}requency\\
\textbf{VHF} & \textbf{V}ery \textbf{H}igh \textbf{F}requency\\


\end{abbreviations}

%----------------------------------------------------------------------------------------
%	SÍMBOLOS
%----------------------------------------------------------------------------------------

\begin{symbols}{lll} % Inclui uma lista de símbolos (uma tabela de três colunas)

%List of Symbols
%
$a$ & distance & \si{\meter} \\
$D$ & dimensão da antena & \si{\meter} \\
$P$ & power & \si{\watt} (\si{\joule\per\second}) \\
$r$ & raio & \si{\meter} \\

%%Símbolo, nome e unidade

\addlinespace % espaçamento para separar os símbolos romanos dos grego

$\varphi$ & ângulo polar & \si{\radian} \\
$\theta$ & azimute & \si{\radian} \\
$\lambda$ & comprimento de onda & \si{\meter} \\


\end{symbols}



%----------------------------------------------------------------------------------------
%	ACRÓNIMOS
%----------------------------------------------------------------------------------------

\newcommand{\listacronymname}{List of Acronyms}
\iflanguage{portuguese}{
\renewcommand{\listacronymname}{Lista de Acr\'onimos}
}

%Use GLS
\glsresetall
\printglossary[title=\listacronymname,type=\acronymtype,style=long]


%----------------------------------------------------------------------------------------
%	ACABOU - BOM TRABALHO
%----------------------------------------------------------------------------------------

\mainmatter % Começar numeração da página com numéros árabes (1,2,3 ...)

