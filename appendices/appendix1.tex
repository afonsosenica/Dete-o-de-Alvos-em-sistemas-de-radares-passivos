% Apêndice A

\chapter{Escreve o título do apêndice} %

\label{AppendixA} % para referenciares o apêndice em alguma parte da tese usa o comando \ref{AppendixA}

As dissertações e outros trabalhos científicos podem conter apêndices ou anexos onde são expostos documentos ou outros materiais que tenham sido usados durante o trabalho, sendo imprescindível que se juntem a ele, mas que, pelo volume, não devem ser introduzidos com o texto por perturbarem a sua harmonia e lógica. São, desta forma, colocados enquanto elemento pós-textual, logo a seguir aos glossários (se existirem) ou à bibliografia. Importa, contudo, compreender o que os distingue um do outro.

Os Apêndices englobam materiais elaborados pelo autor, como conjuntos de gráficos, quadros ou tabelas de dados, eventualmente, traduções de textos, organogramas ou esquemas julgados necessários e referenciados no próprio texto.